Dieses Termpaper befasst sich mit dem Thema Key Escrow. Als Ausgangslage wurden diverse journalistische Berichte, Expertenberichte, Mitteilungen von Bundesbehörden, sowie geschichtliche Aufzeichnungen verwendet. Als erstes wurde eine Übersicht über die vergangenen Diskussionen über Key Escrow gegeben, wobei auffällt, dass die das eigentliche Thema Key Escrow. Im Vergleich zu den 90er Jahren stark an Bedeutung verloren hat. Als zweites wurden die Anwendungsgebiete von Key Escrow analysiert. Auf staatlicher Seite könnte dieses Verfahren sehr gut im Bereich des Nachrichtendienstes eingesetzt werden, aber für diesen Zweck wäre es politisch kaum durchsetzbar. Für den firmeninternen Gebrauch wurden diverse Problematiken im Bereich der verschlüsselten Datenablagen adressiert. Auch dem Clipper Chip, einem exemplarischen Beispiels für Key Escrow ist ein Kapitel gewidmet. Als praktisches Beispiel wurde ein Versuch zum Abhören von verschlüsselter Kommunikation bei bekanntem Schlüssel geplant. Dieser Versuch wurde zwei Mal durchgeführt, nur einmal davon erfolgreich. Als zweites praktisches Beispiel wurden einige bekannte Key Management Tools auf Basis ihrer Funktionen analysiert.
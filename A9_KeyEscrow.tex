\documentclass[a4paper, 10pt, fleqn]{article}

\usepackage[utf8]{inputenc}
\usepackage[T1]{fontenc}
\usepackage{textcomp}
\usepackage{lmodern}
\usepackage[ngerman]{babel}
\usepackage{enumerate}

\usepackage{color}
\usepackage{float}

\usepackage{amsmath}
\usepackage{graphicx}

\usepackage[hyphens]{url}
\usepackage{hyperref}
\usepackage{listings}
\lstset{language=[ansi]C++}

\title{Termpaper - KeyEscrow}
\author{Daniel Foehn, Silvio Stappung, Jonas Hansen}
\date{\today} %Es kann ein bestimmtes Datum eingetragen werden

%macro definitions
\definecolor{shadecolor}{RGB}{200,200,200}
\newcommand{\shadebox}[1]{\par\noindent\colorbox{shadecolor}
{\parbox{\dimexpr\textwidth-2\fboxsep\relax}{#1}}}

\begin{document}
\maketitle
\tableofcontents
\listoffigures
\listoftables
\clearpage
\section{Abstract}
\clearpage
\section{Einleitung}
\clearpage
\section{Aktuelle Situation}
	\subsection{Rechtliche Situation der Schweiz}
		% Silvio
	\subsection{Rechtliche Situation im Aussland}
		% Silvio
	\subsection{Position von Experten}
		% Dani
\clearpage
\section{Anwendungsgebiet}
	\subsection{Staatlich}
		% voranalyse: Silvio
	\subsection{Firmenintern}
		% Jonas
	\subsection{Key Escrow vs. Recovery Agents}
		% Jonas
\clearpage
\section{Praktisches Beispiel}
	\subsection{Abhören von verschlüsselter Kommuniktion bei bekannten Schlüsseln}
	% TODO@16.11.
	\begin{figure}[H]
		\centering
		%\includegraphics[width=.8\textwidth]{""}
		\caption{Darstellung Versuchsaufbau}
		\label{fig:versuchsaufbau}
	\end{figure}
	\shadebox{conf t}
	\shadebox{monitor session 1 source interface Fa0/1}
	\shadebox{monitor session 1 destination interface Fa0/21 encapsulation repliate}
	\subsection{Analyse von bereits vorhandenen Tools} % keine Enterprise-lösungen
\clearpage
\section{Zukunftsaussichten}
\clearpage
\section{Fazit}
\clearpage
\section*{Quellen}h
\url{https://en.wikipedia.org/wiki/Key_escrow}\\
\url{http://www.inside-it.ch/articles/38866}\\
\url{https://www.cdt.org/files/pdfs/paper-key-escrow.pdf}\\
\url{http://security.stackexchange.com/questions/36106/what-is-the-difference-between-key-escrow-and-a-recovery-agent}\\
\url{https://en.wikipedia.org/wiki/Key_management}\\
\url{https://epic.org/crypto/key_escrow/TIS_cke.html}\\
\url{http://www.crypto.com/papers/escrow-acsac11.pdf}\\
\url{https://www.admin.ch/opc/de/federal-gazette/2006/5693.pdf} \\ %Bericht des Bunbdesrates zu einem Postulat der Sicherheitskommision Sicherheitskommision mit Kapitel zu Key Escrow (Kapitel 7) 
\url{https://www.schneier.com/paper-key-escrow.html} \\ %Arbeit zum überblick und Vergleich von Key Recovery, Escrow & Drittverschlöüsselung, sowie zur Geschichte der "Entschlüsselung" 
\url{http://faculty.nps.edu/dedennin/publications/key-concerns.pdf} \\ %Vorstoss eines US Senators für Key Escrow nach dem 11. September 2001
\url{http://www.computerweekly.com/news/2240042808/US-abandons-key-escrow-encryption-plan} \\ %Meldung und Resaktionen zum Rückzug des Vorschlages des Senators
\url{https://netzpolitik.org/2015/vordertuer-statt-hintertuer-nsa-chef-fordert-key-escrow-mit-verteilten-schluesseln/} \\ %NSA Chef fordert die Einführung von Key Escrow
\url{http://www.linux-magazin.de/NEWS/CCC-fuer-Verbot-unverschluesselter-Kommunikation} \\ %CCC fordert Verbot von unverschlüsselter Kommunikation anstelle im Gegensatz zu Politiker, welche eine Aufhebung von verschlüsselungen und/oder eine Escrow-Pflicht verlangen
\url{https://wiki.wireshark.org/ssl}
\end{document}
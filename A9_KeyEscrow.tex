\documentclass[a4paper, 10pt, fleqn]{article}

\usepackage[utf8]{inputenc}
\usepackage[T1]{fontenc}
\usepackage{textcomp}
\usepackage{lmodern}
\usepackage[ngerman]{babel}
\usepackage{enumerate}

\usepackage{amsmath}
\usepackage{graphicx}

\usepackage[hyphens]{url}
\usepackage{hyperref}
\usepackage{listings}
\lstset{language=[ansi]C++}

\title{Termpaper - KeyEscrow}
\author{Daniel Foehn, Silvio Stappung, Jonas Hansen}
\date{\today} %Es kann ein bestimmtes Datum eingetragen werden

\begin{document}
\maketitle
\tableofcontents
\listoffigures
\listoftables
\clearpage
\section{Abstract}
\clearpage
\section{Einleitung}
\clearpage
\section{Aktuelle Situation}
	\subsection{Rechtliche Situation der Schweiz}
	\subsection{Rechtliche Situation im Aussland}
	\subsection{Position von Experten}
	
\clearpage
\section{Anwendungsgebiet}
	\subsection{Staatlich}
	\subsection{Firmenintern}
	\subsection{Key Escrow vs. Recovery Agents}
\clearpage
\section{Praktisches Beispiel}
	\subsection{Abhören von verschlüsselter Kommuniktion bei bekannten Schlüsseln}
\clearpage
\section{Zukunftsaussichten}
\clearpage
\section{Fazit}
\clearpage
\section*{Quellen}
\url{https://en.wikipedia.org/wiki/Key_escrow}\\
\url{http://www.inside-it.ch/articles/38866}\\
\url{https://www.cdt.org/files/pdfs/paper-key-escrow.pdf}\\
\url{http://security.stackexchange.com/questions/36106/what-is-the-difference-between-key-escrow-and-a-recovery-agent}\\
\url{https://en.wikipedia.org/wiki/Key_management}

\end{document}

Das folgende Kapitel bietet einen Einblick in die aktuelle Situation zur Hinterlegung von Schlüsseln, sowie einen Abriss über die allgemeine sicherheitspolitische Lage im Hinblick auf die Kryptografie von elektronischen Daten.
	
	
	\subsection{Rechtliche Situation der Schweiz}
		% Silvio

Die sicherheitspolitische Lage wurde auch in der Schweiz in den vergangenen Jahren nicht einfacher. Trotzdem ist auch in der Schweiz das Hinterlegen von kryptographischen Schlüsseln kein neues Thema. \\
Wie in anderen Ländern ist auch in der Schweiz das Hauptargument für staatlichen Zugang zu kryptografischen Schlüsseln die Bekämpfung des Terrorismus. Auch bereits im Jahr 2005 wurde von der sicherheitspolitischen Kommission des Nationalrates ein Postulat eingereicht, welches unter anderem auch die Dechiffrierung von Kommunikationsverbindungen vereinfachen vorsah. \cite{null} \\ % (KP 7.1. / 7.2.)
Zu diesem Zeitpunkt wurde von staatlicher Seite her vor Allem an der Dechiffrierung von Daten gearbeitet, welche hauptsächlich im Bereich der militärischen Funkaufklärung zu gunsten des strategischen Nachrichtendienstes durchgefüphrt wurde. Da mit der Zeit die Verschlüsselungsalorithmen und auch deren Schlüssel immer stärker und schwerer zu brechen waren, wurde unter anderem Vorgeschlagen, ein Key Escrow System zur Hinterlegung aller kryptografischen Schlüssel zu erstellen und die Verwendung von nicht hinterlegten Schlüsseln unter Strafe zu stellen. Als etwas weniger strenge alternative wurde auch die Hinterlegung der Algorithmen, anstelle der Schlüssel, diskutiert, da dies technisch im Vergleich einfacher umzusetzen wäre. \cite{null} \\ 
\\ %(KP 7.2.)
Die Einführung eines Key Escrow Systemes für die Schweiz wurde schlussendlich vom Bundesrat aufgrund der folgenden Punkte abgelehnt \cite{null}: % KP 7.3

\begin{itemize}
  \item bereits eingeführte kryptografische Systeme müssten ersetzt werden
  \item Politisch nicht durchsetzbar aufgrund der hohen Einführungskosten
  \item zentraler Keystore wäre ein Riskantes Ziel für Cyberangriffe
  \item massive Kosten für Administration und Unterhalt des Systems
\end{itemize}

Zudem würde man auch nach erfolgreicher Einführung eines Key Escrow Systemes vor einigen Problemen stehen. Verstösse gegen das Verschlüsselungsverbot mit nicht hinterleten Schlüsseln wäre hierbei kaum Nachweisbar und somit von Staatlicherseite nicht verfolgbar. \cite{null} \\%KP 7.3.
\\
Auch das neue Nachrichtendienstgesetz, welches voraussichtlich im Jahr 2017 in Kraft treten soll, befasst sich stark mit der Informationsbeschaffung und somit auch mit der Entschlüsselung von Daten. Trotzdem ist  zehn Jahre nach der oben erwähnten Debatte von Key Escrow keine Rede mehr. \cite{null} \\ %Nachrichtendienstgesetz 1 & 2
		
	\subsection{Rechtliche Situation im Aussland}
		% Silvio
	\subsection{Position von Experten}
		% Dani
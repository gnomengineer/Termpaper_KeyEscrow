
Das folgende Kapitel bietet einen Einblick in die rechtliche und geschichtliche Situation zur Hinterlegung von Schlüsseln, sowie einen Abriss über die allgemeine sicherheitspolitische Lage im Hinblick auf die Kryptografie von elektronischen Daten.
	
	
	\subsection{Situation der Schweiz}
		% Silvio

Die sicherheitspolitische Lage wurde auch in der Schweiz in den vergangenen Jahren nicht einfacher. Trotzdem ist auch in der Schweiz das Hinterlegen von kryptographischen Schlüsseln kein neues Thema. \\
Wie in anderen Ländern ist auch in der Schweiz das Hauptargument für staatlichen Zugang zu kryptografischen Schlüsseln die Bekämpfung des Terrorismus. Bereits im Jahr 2005 wurde von der sicherheitspolitischen Kommission des Nationalrates ein Postulat eingereicht, welches unter anderem auch die Dechiffrierung von Kommunikationsverbindungen zu vereinfachen vorsah. \cite[S. 5733f]{adminch} \\ 
Zu diesem Zeitpunkt wurde von staatlicher Seite vor allem an der Dechiffrierung von Daten gearbeitet, welche hauptsächlich im Bereich der militärischen Funkaufklärung zu Gunsten des strategischen Nachrichtendienstes durchgeführt wurde. Da mit der Zeit die Verschlüsselungsalorithmen und auch deren Schlüssel immer stärker und schwerer zu brechen waren, wurde unter anderem Vorgeschlagen, ein Key Escrow System zur Hinterlegung aller kryptografischen Schlüssel zu erstellen und die Verwendung von nicht hinterlegten Schlüsseln unter Strafe zu stellen. Als etwas weniger strenge Alternative wurde auch die Hinterlegung der Algorithmen anstelle der Schlüssel diskutiert, da dies technisch im Vergleich einfacher umzusetzen wäre. \cite[S. 5733f]{adminch} \\ 
\\
Die Einführung eines Key Escrow Systemes für die Schweiz wurde schlussendlich vom Bundesrat aufgrund der folgenden Punkte abgelehnt \cite[S. 5735]{adminch}: 

\begin{itemize}
  \item Bereits eingeführte kryptografische Systeme müssten ersetzt werden
  \item Politisch nicht durchsetzbar aufgrund der hohen Einführungskosten
  \item Zentraler Keystore wäre ein Riskantes Ziel für Cyberangriffe
  \item Massive Kosten für Administration und Unterhalt des Systems
\end{itemize}

Zudem würde man auch nach erfolgreicher Einführung eines Key Escrow Systemes vor einigen Problemen stehen. Verstösse gegen das Verschlüsselungsverbot mit nicht hinterlegten Schlüsseln wäre hierbei kaum Nachweisbar und somit von staatlicher Seite nicht verfolgbar. \cite[S. 5735]{adminch} \\
\\
Auch das neue Nachrichtendienstgesetz, welches voraussichtlich im Jahr 2017 in Kraft treten soll, befasst sich stark mit der Informationsbeschaffung und somit auch mit der Entschlüsselung von Daten. Trotzdem ist  zehn Jahre nach der oben erwähnten Debatte von Key Escrow keine Rede mehr. \cite{botschaftndg} \cite{ndgesetz}  \\
		
	\subsection{Situation im Ausland}
		% Silvio
		
Auch im Ausland ist Key Escrow ein altbekanntes Thema. Bereits Anfangs der Neunzigerjahre wurde in den USA unter Präsident Clinton mehrmals über die Einführung eines solchen Systemes beraten. \cite[S. 5735]{adminch} \cite{denning}
Deshalb ist es auch nicht verwunderlich, dass nach den Anschlägen vom 11. September 2001 von diversen amerikanischen Politikern ein Key Escrow System gefordert wurde. \\
Der wohl einflussreichste Politiker, welche selbige Forderung stellte war der republikanische Senator Judd Gregg, welcher am 13. September 2001 ein neues Regime im Hinblick auf die Verschlüsselung forderte. Judd Gregg forderte dabei die zentralisierte Speicherung aller Schlüssel für in Amerika gebaute sowie importierte kryptografische Produkte. Die Regierung Bush's hat dieses Begehren allerdings abgelehnt und auch keine weiteren Massnahmen in der Verschlüsselungspolitik zur Terrorbekämpfung vorgeschlagen. \cite{denning}\\ 
\\
Doch auch in der heutigen Zeit gibt es immer wieder Forderungen nach staatlicher Einsicht in alle verschlüsselten Verbindungen. Wie früher geschieht dies unter dem Deckmantel der Terrorberkämpfung. Auch in Europa werden solche Diskussionen geführt. Ein Beispiel dafür ist der britische Premierminister David Cameron, der sich im Hintergrund der Anschläge auf das Satieremagazin 'Charlie Hebdo' in Paris von Anfangs Januar 2015 zu diesem Thema äusserte. \\
Cameron forderte ein allgemeines Verbot von jeglicher Kommunikation, welche von staatlicher Seite her nicht mitgelesen werden kann. Allerdings wird diese Forderung von vielen Stellen als starken Angriff auf die Privatsphäre erachtet. \cite{insideit} \\
Auf der anderen Seite zeigte der Fall 'Edward Snowden' auch die Praktiken des Amerikanischen Geheimdienstes NSA sowie dessen britischen Pendant GCHQ auf, wie diese staatlichen Organisationen jegliche Art von Informationen zu entschlüsseln versuchen. Auch der Chef der NSA, Michael Rogers, forderte die Hinterlegung von kryptografischen Schlüsseln zur Entschlüsselung zuhanden der Geheimdienste. Dabei verfolgte er einen etwas anderen Ansatz. Rogers schlug vor, die Schlüssel verteilt bei verschiedenen Stellen zu hinterlegen, welche nur zusammen den eigentlichen Schlüssel generieren können. Dies ist ein bekannter Ansatz, wie er bei vielen grösseren sicherheitsrelevanten Einrichtungen verwendet wird. Diese Escrow Pflicht würde sich allerdings nur auf US-Hersteller auswirken und hätte keinen Einfluss auf ausländische Unternehmungen. \cite{annabiselli} \\
\\
Auch im Ausland lässt sich sagen, dass im Allgemeinen viel über das Thema der staatlichen Einsicht in verschlüsselte Daten diskutiert wird. Allerdings sehen die meisten Staaten aufgrund des Aufwandes, der Kosten sowie der sicherheitstechnischen Bedenken von Key Escrow Systemen ab und wenden sich der herkömmlichen Entschlüsselung der Daten zu.


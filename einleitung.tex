Die Verschlüsselung von Daten hat in den vergangenen Jahren drastisch an Bedeutung gewonnen. Auch durch den technischen Fortschritt konnte Verschlüsselung immer besser auch in der wirtschaftlichen Welt eingesetzt werden. Auch durch die Einführung des asymetrischen Verschlüsselungsverfahren wurde das Hauptproblem, der sichere Schlüsselaustausch gekonnt umgangen. \\
Diese neuen Punkte stellten die Geheimdienste vor grosse Herausforderungen. Der Wunsch der Wirtschaft nach sicherer Kommunikation stand in einem grossen Konflikt mit dem staatlichen Interesse nach Überwachung und Sicherstellung der Staatssicherheit. Verschlüsselung wurde plötzlich nicht mehr nur von IT-Experten verwendet, sondern auch von \textit{normalen} Personen. Ein Grund dafür war sicherlich auch, dass diverse Betriebs- und andere IT-Systeme standartmässig Verschlüsselungen einbauten, sowie dessen Verwendung dem Endbenutzer auch auf sehr einfache Weise zugänglich gemacht wurden. \\
Da natürlich nun auch Terroristen und andere Kriminelle diese Verschlüsselungsmethoden einfacher nutzen konnten stand man von geheimdienstlicher Seite plötzlich vor einem grösseren Problem. Die steigende Masse an kriptografischem Material, welches es zu überprüfen gab, wurde immer Grösser. Aus diesem Grund suchte man eine alternative Möglichkeit, um im Gebrauchsfall an die verschlüsselten Originaldateien kommen könnte. \\
\\
Dieses Termpaper befasst sich hierbei mit einer möglichen Lösung für dieses Problem, dem Key Escrow.\\
Key Escrow beschreibt hierbei die Hinterlegung des kryptografischen Schlüssels bei einer \textit{trusted third party}, einer vertrauesnwürdigen Drittstelle. Dabei ist Key Escrow nicht nur im staatlichen Kontext zu verstehen, sondern auch in der Privatwirtschaft einsetzbar. Auch dort muss es Möglichkeiten geben, beispielweise beim Tode eines Mitarbeiters, an seine verschlüsselten Arbeitsdaten zu gelangen. \\
\\
Dieses Termpaper gibt hierbei einen Einblick in die geschichtliche Entwicklung von Key Escrow und dessen Anwenungsdiskusionen, die Anwenungsmögichkeiten sowie einen kleinen Feldversuch für den Einsatz von hinterlegten Schlüsseln.

 % Problematiken
 % KEY ESCROW!!!!!!!!!!
Das folgende Kapitel zeigt mögliche Anwendungen von Key Escrow Systemen und deren Alternativen für den Gebrauch von staatlichen Behördern, sowie in der Privatwirtschaft auf.
	
	\subsection{Staatlich}
		% Silvio
Die immer stärker werdenden kryptografischen Verfahren stellen die staatlichen Sicherheitsbehördern vor immer grössere Probleme. Die Anwendung von Verschlüsselung durch Straftäter erschwert die Arbeit der Sicherheutsbehörder enorm. \\
Nach terroristischen Anschlägen werden immer wieder Stimmen laut, welche die Einführung einer Möglichkeit zur Erleichterung der Entschlüsselung von Kommunikationsdaten fordern \cite{CAMERON} \cite{NSA} \cite{GREGG}. 
Dabei werden immer wieder die selbigen Themen angesprochen:

	\subsubsection{Erstellung eines Key Escrow Systemes}
Die Einführung eines Key Escrow Systemes wäre für jeden Geheimdienst wohl das höchste aller Gefühle. Allerdings hat ein staatliches Key Escrow System politisch einen sehr schweren Standpunkt. Dies liegt vor allem daran, dass ein solches System zu einem gewaltigen Verlust, wenn nicht gerade eine totale Aufgabe der Privatsphäre zur Folge hätte und die Gesellschaft kaum mit einer so drastischen Massnahme einverstanden wäre.\cite {CAMERON} \\
Fraglich wäre auch der Nutzen des Sytems im Hinsicht auf grössere kriminelle Organisationen. Zu glauben, dass alle Kriminellen Verschlüsselungstools benutzen würden, für welche der Schlüssel im Escrow System hinterlegt ist wäre sehr fraglich. Beispielsweise hat die Terrororganisation Al-Qaida von der Organisation nahestehenden Kryptologen ein eigenes Verschlüsselungssystem entwickeln lassen \cite{GREGG}. Dass es sich dabei nicht im einen Einzelfall handelt, lasst sich auch stark annehmen.
	
	\subsubsection{Hinterlegung von verwendeten Verschlüsselungsalgorithmen}
Eine nicht so schwerwiegende Alternative zu Key Escrow wäre die Hinterlegung des Verschlüsselungsverfahrens ohne Hinterlegung des verwendeten kryptografischen Schlüssels. Diese Hinterlegung sollte zum Nutzen haben, dass den Strafverfolgungsbehörden die Arbeit etwas erleichtert würde. Dies hätte allerdings auch nur dann einen Nutzen, wenn nur schwächere oder Algorithmen mit einer Sicherheitslücke verwendet werden dürfen. Diese Lücken wären somit auch ein Eigentor für die Sicherheitsbehörden, da sie somit Kriminellen eine weiter Angriffsfläche auf die Bürger bieten würden. \cite{BUND KP. 7.3.}

	
	\subsubsection{Verbot von alternativen Verschlüsselung}
Mit der Einführung eines Escrow Systemes oder der Einführung einer Hinterlegungspflicht für Algorithmen würde auch dazu führen, dass die Gegenseite, das heisst die Verwendung nicht hinterlegter Schlüssel / Algorithmen, verboten werden müsste. Auf der einen Seite wäre ein solches Verbot sehr schwer durchzusetzten, geschweige überhaupt zu kontrollieren, auf der andern Seite wäre es von der Strafnorm her auch kaum möglich, eine genug hohe Maximalstrafe ansetzen zu können, um grössere kriminelle Organisationen von der illealgen Verwendung der Verschlüsselung abzuwenden. \cite{BUND KP 7.3.} \\
Zudem wäre ein solches Verbot auch nur dann vernünftig brauchbar und durchsetzbar, wenn es von der internationalen Gemeinschaft getragen und somit in allen Ländern durchgesetzt würde \cite{GREGG}. Da der Internationale Weg in Sache Verschlüsselungspolitik in eine andere Richtung weht, zeigt sich darin, dass sich einige Länder, welche in den neunziger Jahren die Kryptografie noch eingeschränkt hatten, in Richtung einer offenen Kryptografiepolitik bewegen. \cite{EPIC} \cite{BUND 7.2.3} \\
Eine Einführung eines solchen Verbotes würde einen Staat Interational stark ins Abseits setzten und das Verbot wäre somit auch kaum duchsetzbar.
		
	\subsection{Firmenintern}
		% Jonas
	\subsection{Key Escrow vs. Recovery Agents}
		% Jonas